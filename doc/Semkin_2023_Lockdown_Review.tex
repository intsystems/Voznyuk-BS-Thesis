\input{Preambula}
\usepackage[backend=biber]{biblatex}

\addbibresource{Semkin_2023_Lockdown.bib}

\title{Исследование распространения эпидемий в графовой модели SIR и влияния локдауна на динамику заболеваемости популяции}
\author{Сёмкин К. \and Бишук А.}
\date{}

\begin{document}
	
	\maketitle
	
	\begin{abstract}
		Задача создания и обоснования математических моделей распространения эпидемий всегда являлась и будет являться актуальной в наше время. Существуют оправданные и реалистичные модели распространения, такие как SIR и SIER, основанные на дифференциальных уравнениях динамики количества больных и здоровых жителей, но их область применимости ограничивается большими масштабами наблюдения. В данной работе рассматривается аналог SIR, основанный на графе контактов, для анализа заболеваемости на малых мастштабах (предприятие, небольшая коммуна), исследуется вероятностная динамика распространения эпидемии в зависимости от параметров модели, а также влияние разных ограничительных мер. В частности интересно обоснование противоречивого эффекта, связанного с ростом заболевших при введении локдауна. Теоретические результаты демонстрируются проведением численных экспериментов посредством симулирования эпидемии на графе контактов, а также на основе реальных данных.
	\end{abstract}
	
	\section*{Введение}
	
	В связи с известными событиями интерес к эпидемиологии и её методам сильно вырос. Для понимания динамики протекания короновирусной инфекции в 2020 году в разных местах Земли и на разных масштабах, а также для подготовки к будущим вспышкам заболеваемости, как никогда актуально построение адекватных математических моделей развития болезней в людских популяциях \cite{COLIZZA2007364}. Классические техники моделирования эпидемий опираются на параметризованные автономные системы дифференциальных уравнений, описывающие динамику изменения количества болеющих и здоровых людей. Эти модели дают хорошее понимание протекания болезни на больших масштабах (города, страны), но не способны описывать заболевание в небольших общественных структурах, например, промышленное предприятие, небольшую деревню или студенческое общежитие. В данной работе исследуется графовый подход к моделированию распространения инфекции, а именно вводится \textit{граф контактов}, по которому болезнь может <<кочевать>>. В качестве представления болезни используется стандартная для эпидемиологии модель SIR/SEIR \cite{seirsplus}, в которой каждому человеку (вершине в графе) сопоставляется некоторое состояние (больной, здоровый и т.д.), после чего в дискретном времени происходят смены этих состояний с некоторыми вероятностями и система эволюционирует.
	
	Также исследуются различные эффекты от мер по борьбе с инфекцией, таких как тестирование, изоляция и, самое интересное, \textit{локдаун}. Именно ему уделяется основное внимание, так как его введение может привести к необычному последствию --- росту заболеваемости среди населения. Но обнаружить такое поведение в стандартных моделях не представляется возможным, поэтому цель данной работы --- найти условия возникновения такого эффекта в модели и продемонстрировать его на численных экспериментах. 
	
	Изучение эпидемий на больших популяциях позволяет моделировать этот процесс в среднем, и даже получать точные аналитические решения \cite{harko2014exact}. В зависимости от поставленной прикладной задачи возникает необходимость моделировать процесс эпидемии с разной степенью подробности. Так, например, простейшая модель SI \cite{allen1994some} рассматривает всего два состояния: больной и здоровый. В этой модели не рассматривается формирование иммунитета: здоровый всегда может заразиться при контакте с инфекцией. Существуют модели, рассматривающие дополнительно формирование иммунитета, инкубационный период, летальные исходы и многие другие возможные состояния. Одной из таких моделей является SEIR(S) \cite{capasso2008mathematical}. Моделирование в среднем не подходит для небольших или слишком разнородных популяций. Эту проблему позволяют решить модели распространения эпидемии на графах \cite{moreno2002epidemic}, \cite{pastor2015epidemic}. Распространение эпидемии на графе контактов можно рассматривать, например, при помощи цепи Маркова \cite{gomez2010discrete}. Однако моделирование распространения болезни на больших графах со сложной структурой имеет высокую алгоритмическую сложность. Наиболее распространенной является задача прогнозирования течения эпидемии \cite{leitch2019toward} и оценка индивидуальных рисков. Результаты изучения распространения эпидемии на графах	могут быть использованы не только для анализа заболеваний. Например, распространение слухов или автомобильного трафика можно описать схожим математическим аппаратом \cite{de2013anatomy}. Фундаментом для данной статьи является \cite{base_article}, где, в частности, введена модель болезни на графе и где в её рамках исследован эффект локдауна.
	
	В работе ставится задача обобщить модель из \cite{base_article}, сформулировать новые условия возникновения роста заболеваемости при введении карантина и явно показать этот эффект в численном эксперименте. Т.о. появится возможность испытывать обновлённую модель в более широком спектре реальных ситуаций, а также пересмотреть локдаун как однозначно позитивную меру противодействия эпидемии.
	
	\section*{Постановка задачи}
	
	\subsection*{Формальная часть}
	
	Формально задача состоит в выявлении зависимости роста заболеваемости при введении локдауна от графа контактов и параметров динамики развития болезни.
	
	Пусть $ G = (V, E) $ --- исходный граф контактов, $ G^q = (V^q, E^q) $ --- граф контактов при введении карантинного режима. Два графа строятся на одних и тех же вершинах. $ G $ можно интерпретировать как контакты людей в рабочее время, граф имеет произвольную структуру, но является плотным. $G^q$ должен представлять изоляцию вершин друг от друга, поэтому это граф состоящий из множества клик небольшого относительно $\lvert V \rvert$ размера. Ребро этих графов соответствует контакту данных лиц, а вес ребра $ \beta_{ij} $ соответствует вероятности вершины заразиться, если её сосед сам находится в состоянии \textit{Infected (I)}. \footnote{Вот здесь хорошо было бы написать про формальные состояния вершин}
	
	Будем понимать под $ G_t $ граф контактов на дискретном временном шаге $ t $, т.е. его графовую структуру, а также состояния каждой вершины в данный момент. Под $ I(G) $ будем понимать множество больных вершин в графе (или кол-во больных в графе, в зависимости от контекста). 
	
	Т.о. задача состоит в поиске условий на $ t_0 $, $ G $ и $ G_q $, при которых $ \underset{t \ge t_0}{\max} \, I(G^q_t) \ge \underset{t \ge t_0}{\max} \, I(G_t) $, где $ t_0 $ момент введения локдауна. Другая возможная постановка: найти условия на те же параметры, при которых $ \beta_{ij}^q \ge \beta_{ij} $.
	
	\subsection*{Метод эксперимента}
	
	Опишем постановку \textit{экспериментальной части}: создаётся два графа контактов на $ N $ вершинах со взвешенными рёбрами. Вес любого ребра $ w_{(a, b)} \in [0, 1] $ и интерпретируется как доля времени, проведённая вершиной $ a $ с вершиной $ b $ за всё время на данном графе контактов \footnote{то есть делим на все время что граф существует? или на все время что a и b в принципе взаимодействуют, не обязательно в текущее взаимодействие.}. Задаются гиперпараметры эпидемии: вероятности перехода между подверженным/больным/выздоровевшим для вершины. Задаётся начальное распределение больных/здоровых вершин. Далее, вне локдауна, одна итерация для эпидемии проходит так: сначала активен <<рабочий>> граф, в котором согласно вычисленным вероятностям вершины меняют состояния. Далее становится активен <<домашний>> граф, на котором происходит аналогичные действия. При введении карантина же одна итерация будет происходить два раза на <<домашнем>> графе. По таким правилам эпидемия эволюционирует любое заданное время с сохранением истории состояний для каждой вершины.
	
	\section*{Вычислительный эксперимент}
	
	Главная задача вычислительного эксперимента --- проиллюстрировать эффект локдауна, сэмулировав эпидемию на графах контактов, в которых такой эффект вообще возможен. Т.о. данная демонстрация служит подтверждением неголословности оговоренного ранее.
	
	\subsection*{Программный пакет SEIRSplus}
	
	Для начала были проведены симулирования заболеваемости с помощью библиотеки SEIRSplus \cite{seirsplus}, которая предоставляет богатые средства инициализации модели, а также её тонкой настройки в любой момент развития эпидемии. Стандартный граф контактов генерировался на заданном наборе вершин $ V = \{1, \ldots, N\} $ и содержал случайное количество рёбер, которое было точно больше половины числа ребёр в полносвязном графе. Граф карантина же собирался из клик случайного размера от 1 до 5 вершин. Чувствительность к заболеванию генерировалась случайно для каждой вершины, со средним вокруг значения $ 0.5 $, вероятность восстановления у заболевших вершин была одна для всех $ 0.3 $. 
	
	В итоге запускалось две симуляции, в одной из которых карантин не вводился, а в другой вводился на некоторое время. Результаты приведены на рис. \ref{pic:basic}. Здесь представлена динамика количества заражённых узлов для размера популяции $ 10, 100, 500 $ вершин в графе. Синия линия --- развитие болезни без локдауна, красная и розовая --- развитие с локдауном, где розовая линия как раз отвечает периоду изоляции. Зелёными линиями обозначены точки входа и выхода из локдауна.
	
	К сожалению, данная библиотека хоть и обладает огромным потенциалом, но всё же является технически недоработанной, а также скрывает в себе некоторые нежелательные методы ускорения сэмплирования, что не является приемлемым для текущего исследования. 
	
	\begin{figure}[h]
		\begin{minipage}{0.49\linewidth}
			\includegraphics[width=\linewidth, keepaspectratio]{../figs/basic_experiment_small_population}
			
			\centering
			Small population
		\end{minipage}
		\begin{minipage}{0.49\linewidth}
			\includegraphics[width=\linewidth, keepaspectratio]{../figs/basic_experiment_medium_population}
			
			\centering
			Medium population
		\end{minipage}
		\begin{center}
			\begin{minipage}{0.5\linewidth}
			\includegraphics[width=\linewidth, keepaspectratio]{../figs/basic_experiment_big_population}
			
			\centering
			Large population
		\end{minipage}
		\end{center}
		\caption{Графики количества заражённых в двух режимах протекания эпидемии от времени}\label{pic:basic}
	\end{figure}

	\subsection*{Симуляция распространении эпидемии}
	
	В итоге для данной работы был составлен собственный программный пакет для создания эпидемий и проведения сэмплирования на них. Эпидемия проходит на двух графах контактов, отвечающих за <<рабочий>> и <<домашний>> режимы для вершин, во времени, в которой элементарная единица есть один день. Причём если не введён локдаун, то половину дня эпидемия проводится на одном графе, а половину на другом. Вершины меняют состояния согласно вероятностям, заданным по правилам модели. Также задаются временные рамки, в течении которых вводится локдаун, который симулируется как проведение эпидемии только на <<домашнем>> графе.
	
	Главная цель вычислительного эксперимента --- проиллюстрировать эффект локдауна для эпидемий и графов с параметрами, для которых этот самый эффект обязан проявиться согласно теоретическим оценкам. 
	
	Результаты сэмплирований для нескольких графов представлены на рис. \ref{pic:evidence_1} и рис. \ref{pic:evidence_2}. Основные графы представляли собой полносвязанный и полный цикл на 10 и 15 вершинах, графы карантина представляли собой совокупность клик размера 3 и 4 соответственно. Каждая 3 и каждая 4 вершина в графах были изначально заражены. После проведения многих опытов по описанной выше схеме и усреднения были получены треки эволюции заболеваемости, в которых сравниваются количества заболевших в эпидемии с карантином и идущей параллельно от неё (до ограничения болезни развиваются одинаково) без карантина. Из данных графиков ясно видно, что для данных конфигураций эпидемия проходит тяжелее во время локдауна и некоторое время после него.
	
	% evidence 1
	\begin{figure}[h]
		\begin{center}
			\begin{minipage}{0.49\linewidth}
			\includegraphics[width=\linewidth, keepaspectratio]{../figs/evidence2/init}
			
			\centering
			Начальная конфигурация вершин и вид <<рабочего>> графа
		\end{minipage}
		\begin{minipage}{0.49\linewidth}
			\includegraphics[width=\linewidth, keepaspectratio]{../figs/evidence2/start_with_ld}
			
			\centering
			Вид <<домашнего>> графа
		\end{minipage}
		\end{center}
	
		\begin{center}
			\begin{minipage}{0.49\linewidth}
				\includegraphics[width=\linewidth, keepaspectratio]{../figs/evidence2/tracks}
				
				\centering
				Сравнительные треки эпидемии с локдауном и без него
			\end{minipage}
		\end{center}
	
		\caption{Визуализация протекания эпидемии для полносвязанного графа}\label{pic:evidence_1}
	\end{figure}

	% evidence 2
	\begin{figure}[h]
		\begin{center}
			\begin{minipage}{0.49\linewidth}
				\includegraphics[width=\linewidth, keepaspectratio]{../figs/evidence3/init}
				
				\centering
				Начальная конфигурация вершин и вид <<рабочего>> графа
			\end{minipage}
			\begin{minipage}{0.49\linewidth}
				\includegraphics[width=\linewidth, keepaspectratio]{../figs/evidence3/start_with_ld}
				
				\centering
				Вид <<домашнего>> графа
			\end{minipage}
		\end{center}
		
		\begin{center}
			\begin{minipage}{0.49\linewidth}
				\includegraphics[width=\linewidth, keepaspectratio]{../figs/evidence3/tracks}
				
				\centering
				Сравнительные треки эпидемии с локдауном и без него
			\end{minipage}
		\end{center}
		
		\caption{Визуализация протекания эпидемии для графа-цикла}\label{pic:evidence_2}
	\end{figure}

	\printbibliography
	
\end{document}