\section{Заключение}

В данной работе были рассмотрены различные задачи, связанные с детекцией машинно-сгенерированных текстов. Была рассмотрена задача бинарной детекции, когда необходимо определить автора всего документа. В качестве вычислительного эксперимента сравнивалось результаты дообучения различных энкодеров на датасете с русскими текстами.

Другой задачей является детекция смены авторов по фиксированным позициями, например по параграфами или предложениям. Предложено два подхода, первый из которых рассматривает каждый фрагмент как отдельный текст и сводит задачу к предыдущей. Второй подход предлагает использовать марковские линейные цепочки для учета контекста фрагментов. Учет контекста помогает немного улучшить результаты детекции. Кроме того, было показано, что детекция по параграфам позволяет детектору быть более уверенным в своих предсказания, нежели при детекции по предложениям, что подтверждает выводы в других работах о том, что слишком маленькие тексты могут ухудшать качество работы детектора.

Наконец, последней подзадачей является задача поиска единственной смены стиля, когда известно, что сначала идет человеческий текст, а потом идет машинный текст, но при этом смена авторов может быть в произвольной позиции текста. Для данной подзадачи предлагается опять использовать предобученные модели энкодеров. Методы, предложенные для решения этой задачи, на данный момент показывают наилучшие результаты по итогам соревнования SemEval2024~Task 8, посвященному решению этой задачи.