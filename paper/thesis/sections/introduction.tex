\section{Введение}

\textbf{Актуальность темы.} В настоящий момент большие языковые модели стали повсеместно используемым инструментом для решения различных задач. Это стало особенно заметно после выхода чата-интерфейса ChatGPT~\cite{openai2023gpt4} от компании OpenAI. Модель, на основе которой работал чат, значительно превосходила лучшие на тот открытые модели, и тексты, которая она генерировала, стали практически неотличимы от человеческого. Часто даже людям может быть сложно отличить человеческий текст от машинно-сгенерированного~\cite{Dugan2022RealOF}. По этой причине ChatGPT и другие продвинутые языковые модели все чаще стали использоваться для решения повседневных, рабочих и учебных задач. Это, в свою очередь повлекло массовое использование сгенерированных текстов там, где ожидается текст, написанный человеком - например, в домашних заданиях и эссе~\cite{liu2023argugpt}, в текстах выпускных работ и научных статьях~\cite{ma2023ai, liang2024mapping}. Кроме того, часто языковые модели используются для генерации фейковых новостей~\cite{tweepfake2021, loth2024blessing}. Большие языковые модели продолжают развиваться, их качество растет, поэтому должны развиваться и детекторы, с помощью которых можно было бы выделять инородные по стилю фрагменты. Задачу детекции машинно-сгенерированных текстов часто формулируют как задачу бинарной классификации автора текста. Однако при реальном использовании больших языковых моделей, часто ответы модели смешивают вместе с фрагментами, написанными человеком. Получается довольно сложная для решения задача, потому что:
\begin{itemize}
    \item Сгенерированные фрагменты могут быть в произвольных позициях в документе.
    \item Сгенерированных фрагментов может быть сколько угодно много и они могут быть произвольной длины.
    \item При генерации могут использоваться несколько больших языковых моделей.
\end{itemize}
Так как общая задача пока слишком сложная для решения, рассматриваются частные случаи этой задачи. Во-первых, можно ограничивать позиции документов, например разрешить смену автора только по предложениям или по параграфам. Во-вторых, можно разрешить только одну смену авторов, но в произвольном месте в документе. Также, не во всех работах рассматривается устойчивость детекторов к смене доменов или же к смене генерирующей модели и часто рассматривается только лишь один домен, чаще всего научный, новостной или домен студенческих эссе. \\
\textbf{Цели работы.}
\begin{enumerate}
    \item Предложить метод детекции фрагментов различного авторства в документах в случае смены авторов по фиксированным позициям с помощью марковской линейной цепочки.
    \item Предложить метод детекции фрагментов различного авторства в документах в случае единственной смены авторов~--- смены стиля.
\end{enumerate}
\textbf{Практическая значимость.} Предложенные в работе методы могут использоваться для создания моделей детекции для поиска сгенерированного текста в реальных текстах, а также для измерения качества сгенерированного текста при обучении новых языковых моделей.




