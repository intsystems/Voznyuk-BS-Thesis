\begin{center}
    \Large{\textbf{Аннотация}}
\end{center}
В данной работе рассматривается задача детекции фрагментов машинно-сгенерированного в документе. Впервые задача детекции синтетического текста  была поставлена как задача обработки естественного языка в 2019, вскоре после выхода первой открытой языковой модели GPT-2. В такой постановке задача является совокупностью двух различных задач, а именно, выделение фрагментов, различного авторства на основе смены стилистики текста. и его последующая классификация. 


Цель данной работы состоит в разработке базовой модели нейронной сети для классификации искусственных фрагментов. Для реализации этого был проведен анализ существующих методов классификации текстов, изучены различные варианты архитектур нейронных сетей, пригодных для решения поставленной задачи.


В результате работы была разработана рабочая модель классификации, основанная на большой предобученной языковой модели типа трансформер. Для обучения и тестирования модели была использована специально подготовленная и размеченная выборка текстов, содержащих фрагменты, сгенерированные языковыми моделями. В работе подробно описывается процесс сбора и подготовки данных, процесс обучения моделей в различных вариантах нахождения фрагментов в документе, а также приводятся подробно результаты всех поставленных вычислительных экспериментов.

