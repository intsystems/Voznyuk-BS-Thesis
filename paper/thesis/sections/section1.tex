\section{Постановка задачи}
\label{sec1}
Определим документ как конечную последовательность символов из заданного алфавита $\mathbf{W}$. Тогда пространство документов определено как:
$$\mathbb{D} = \Bigl\{\Big[t_j\Big]_{j=1}^n \quad|\quad t_j \in \mathbf{W}, n \in \mathbb{N}\Bigr\}.$$
Дан набор документов $\mathbf{D}$:
$$\mathbf{D} = \bigcup_{i=1}^{N}D^i, D^i \in \mathbb{D}. $$
Пусть известно, что для создания текстов в наборе $\mathbf{D}$ принимали участие $k$ авторов, причем один из авторов обязательно человек. Определим множество авторов:
$$\mathbf{C} = \{0, \dots, k - 1\}.$$
При классификации будем помечать человеческий текст классом 0.

\subsection{Детекция автора всего документа}
\label{binary_detection}

Первой подзадачей является определение автора всего документа. Это задача классификации, в случае бинарной классификации необходимо определить, кто автор текста~--- человек или языковая модель. В случае многоклассовой задачи, автором текста может быть не только человек, но и какая-то модель из заранее известного набора языковых моделей. Формально, детектор определяется как
\begin{align}
    \mathbf{\phi}: \mathbb{D} \rightarrow  \mathbf{C},
\end{align}
где $\mathbf{C} = \{0,1\}$ для бинарной детекции
или $\mathbf{C} = \{0, \dots, k - 1\}$ для многоклассовой детекции и $k$ языковых моделей-авторов. Так как это задача классификации, то метрикой качества здесь может быть точность в случае бинарной классификации и F1-мера в случае многоклассовой классификации.



\subsection{Детекция фрагментов}
\label{fragment_detection}

Следующей подзадачей является нахождение в документе фрагментов другого авторства. Чаще всего подразумевается, что изначально текст написан человеком и в него добавлены фрагменты, написанные одной или несколькими языковыми моделями. Поэтому в таком случае, необходимо сначала разбить документ на фрагменты разного авторства, а после этого для каждого фрагмента необходимо определить его автора с помощью классификации, описанной в \ref{binary_detection}.
Формально, для каждого документа $ \mathbf{D} \in \mathbb{D}$ существует представление
$$ \mathbb{T} = \Bigl\{\Big[t_{s_j}, t_{f_j}, C_j\Big]_{j = 1}^{J} \quad|\quad t_{s_j} = t_{f_{j - 1}},\quad s_j \in \mathbb{N}_0,\quad f_j \in \mathbb{N}, \quad C_j \in  \mathbf{C} \Bigr\},$$
где $J$ - количество фрагментов разного авторства, $t_{s_j}$ и $t_{f_j}$ - начало и конец $j$-ого фрагмента, внутри  которого все токены одного авторства. 
В такой постановке предлагаемая модель детекции описывается композицией отображений
\begin{align}
   \mathbf{\phi}: \mathbb{D} \rightarrow \mathbb{T} \quad \quad \mathbf{\phi} : \mathbf{g} \circ \mathbf{f},
\end{align}
где $\mathbf{f}$~--- отображение для выделения текстовых фрагментов, а $\mathbf{g}$  - отображение для классификации получившихся фрагментов.
В данном случае, метрикой может быть процент пересечения предсказанных фрагментов с истинной разметкой фрагментов вместе с проверкой авторства.


\subsection{Детекция смены стиля}
\label{style_change}
Наконец,  опишем частный случай задачи детекции фрагментов, когда известно, что смена авторов происходит единожды и причем смена авторства строго с человеческого текста на машинно-сгенерированный. Т.е. для документа $d \in \mathbb{D}$  известно, что

$$\exists I \in \mathbf{N}_0 \quad \mathbf{g}([t_{0}, t_{I})) = 0, \quad \mathbf{g}([t_{I + 1},  t_{|d|})) =1,  \quad  0 \leq I < |d|,$$
В таком случае, задача сводится только к нахождению индекса единственного токена, где  происходит смена автора с помощью отображения $\mathbf{f}$. 
Метрикой качества детектора для такой задачи может служить значение ошибки предсказания положения индекса. Предлагается использовать метрику средней абсолютной ошибки, которая для набора документов считает среднее значение модуля разности между истинным положением индекса и предсказанным. 
